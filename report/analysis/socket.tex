\subsubsection{Client}
\paragraph{\textbf{Khởi tạo Winsock}}
\begin{itemize}
    \item \textbf{Prototype:} \texttt{bool initializeWinsock(WSADATA\& wsaData);}
    
    \item \textbf{Mục đích:} Khởi tạo thư viện Winsock để sử dụng các API mạng.
    
    \item \textbf{Cách thức hoạt động:} 
    \begin{enumerate}
        \item Gọi hàm \texttt{WSAStartup()} để khởi tạo Winsock.
        \item Kiểm tra lỗi nếu việc khởi tạo thất bại.
    \end{enumerate}
    
    \item \textbf{Tham số truyền vào:} 
    \begin{itemize}
        \item \texttt{WSADATA\& wsaData}: Một cấu trúc chứa thông tin về phiên bản Winsock được khởi tạo.
    \end{itemize}
    
    \item \textbf{Tham số trả về:} \texttt{true} nếu khởi tạo thành công, \texttt{false} nếu thất bại.
\end{itemize}

\paragraph{\textbf{Nhập địa chỉ IP server}}
\begin{itemize}
    \item \textbf{Prototype:} \texttt{std::string getServerAddressInput();}
    
    \item \textbf{Mục đích:} Nhận địa chỉ IP của server từ người dùng.
    
    \item \textbf{Cách thức hoạt động:} 
    \begin{enumerate}
        \item Hiển thị lời nhắc yêu cầu người dùng nhập địa chỉ IP của server.
        \item Đọc địa chỉ IP từ bàn phím.
    \end{enumerate}
    
    \item \textbf{Tham số truyền vào:} Không có.
    
    \item \textbf{Tham số trả về:} Chuỗi chứa địa chỉ IP của server.
\end{itemize}

\paragraph{\textbf{Lấy thông tin địa chỉ server}}
\begin{itemize}
    \item \textbf{Prototype:} \texttt{addrinfo* getServerAddress(const std::string\& serverAddress, const std::string\& port, const addrinfo\& hints);}
    
    \item \textbf{Mục đích:} Lấy thông tin địa chỉ của server từ hệ thống.
    
    \item \textbf{Cách thức hoạt động:} 
    \begin{enumerate}
        \item Gọi hàm \texttt{getaddrinfo()} để lấy danh sách địa chỉ phù hợp với cấu hình.
        \item Kiểm tra lỗi và trả về danh sách địa chỉ hoặc thông báo lỗi nếu có.
    \end{enumerate}
    
    \item \textbf{Tham số truyền vào:} 
    \begin{itemize}
        \item \texttt{serverAddress}: Địa chỉ IP của server.
        \item \texttt{port}: Cổng mà server sử dụng.
        \item \texttt{hints}: Cấu hình tìm kiếm thông tin địa chỉ (ví dụ: loại giao thức, kiểu socket).
    \end{itemize}
    
    \item \textbf{Tham số trả về:} Con trỏ đến danh sách thông tin địa chỉ \texttt{addrinfo*}, hoặc \texttt{NULL} nếu thất bại.
\end{itemize}

\paragraph{\textbf{Kết nối đến server}}
\begin{itemize}
    \item \textbf{Prototype:} \texttt{bool connectToServer(SOCKET\& ConnectSocket, addrinfo* result);}
    
    \item \textbf{Mục đích:} Thiết lập kết nối từ Client đến Server.
    
    \item \textbf{Cách thức hoạt động:}
    \begin{enumerate}
        \item Lặp qua các địa chỉ trong danh sách \texttt{result}.
        \item Tạo socket với các thông tin địa chỉ phù hợp bằng hàm \texttt{socket()}.
        \item Thực hiện kết nối đến server bằng hàm \texttt{connect()}.
        \item Đóng socket nếu kết nối không thành công và thử địa chỉ tiếp theo.
    \end{enumerate}
    
    \item \textbf{Tham số truyền vào:}
    \begin{itemize}
        \item \texttt{ConnectSocket}: Socket sẽ được sử dụng để kết nối.
        \item \texttt{result}: Danh sách địa chỉ và cổng của server (từ hàm \texttt{getServerAddress()}).
    \end{itemize}
    
    \item \textbf{Tham số trả về:} \texttt{true} nếu kết nối thành công, \texttt{false} nếu thất bại.
\end{itemize}

\paragraph{\textbf{Gửi yêu cầu từ client}}
\begin{itemize}
    \item \textbf{Prototype:}\\ \texttt{bool sendClientRequest(SOCKET\& ConnectSocket, const std::string\& request);}
    
    \item \textbf{Mục đích:} Gửi yêu cầu từ client đến server qua socket.
    
    \item \textbf{Cách thức hoạt động:} 
    \begin{enumerate}
        \item Gửi chuỗi yêu cầu bằng hàm \texttt{send()}.
        \item Kiểm tra lỗi nếu việc gửi thất bại và hiển thị thông báo lỗi.
    \end{enumerate}
    
    \item \textbf{Tham số truyền vào:} 
    \begin{itemize}
        \item \texttt{ConnectSocket}: Socket kết nối đến server.
        \item \texttt{request}: Chuỗi yêu cầu cần gửi.
    \end{itemize}
    
    \item \textbf{Tham số trả về:} 
    \begin{itemize}
        \item \texttt{true} nếu gửi thành công.
        \item \texttt{false} nếu thất bại.
    \end{itemize}
\end{itemize}

\paragraph{\textbf{Nhận loại phản hồi từ server}}
\begin{itemize}
    \item \textbf{Prototype:} \texttt{std::string receiveResponseType(SOCKET\& ConnectSocket, char* recvbuf, int recvbuflen);}
    
    \item \textbf{Mục đích:} Xác định loại phản hồi từ server (văn bản hoặc tệp tin).
    
    \item \textbf{Cách thức hoạt động:} 
    \begin{enumerate}
        \item Nhận kích thước loại phản hồi qua \texttt{recv()}.
        \item Nhận dữ liệu mô tả loại phản hồi từ server.
        \item Kiểm tra lỗi trong quá trình nhận.
    \end{enumerate}
    
    \item \textbf{Tham số truyền vào:} 
    \begin{itemize}
        \item \texttt{ConnectSocket}: Socket kết nối đến server.
        \item \texttt{recvbuf}: Bộ đệm để nhận dữ liệu.
        \item \texttt{recvbuflen}: Kích thước bộ đệm.
    \end{itemize}
    
    \item \textbf{Tham số trả về:} Chuỗi xác định loại phản hồi (\texttt{"text"} hoặc \texttt{"file"}).
\end{itemize}

\paragraph{\textbf{Xử lý phản hồi dạng tệp}}
\begin{itemize}
    \item \textbf{Prototype:} \texttt{std::string handleFileResponse(SOCKET\& ConnectSocket, char* recvbuf, int recvbuflen);}
    
    \item \textbf{Mục đích:} Nhận và lưu trữ tệp tin được gửi từ server.
    
    \item \textbf{Cách thức hoạt động:} 
    \begin{enumerate}
        \item Nhận tên tệp và kích thước tệp từ server.
        \item Mở tệp để ghi dữ liệu.
        \item Nhận từng phần dữ liệu từ server và ghi vào tệp.
        \item Hiển thị thanh tiến trình trong quá trình nhận tệp.
        \item Đóng tệp sau khi hoàn tất.
    \end{enumerate}
    
    \item \textbf{Tham số truyền vào:} 
    \begin{itemize}
        \item \texttt{ConnectSocket}: Socket kết nối đến server.
        \item \texttt{recvbuf}: Bộ đệm để nhận dữ liệu.
        \item \texttt{recvbuflen}: Kích thước bộ đệm.
    \end{itemize}
    
    \item \textbf{Tham số trả về:} Tên của tệp tin đã nhận.
\end{itemize}

\paragraph{\textbf{Xử lý phản hồi dạng văn bản}}
\begin{itemize}
    \item \textbf{Prototype:} \texttt{std::string handleTextResponse(SOCKET\& ConnectSocket, char* recvbuf, int recvbuflen);}
    
    \item \textbf{Mục đích:} Nhận phản hồi dạng văn bản từ server và in ra màn hình.
    
    \item \textbf{Cách thức hoạt động:} 
    \begin{enumerate}
        \item Nhận kích thước phản hồi từ server.
        \item Nhận từng phần của phản hồi và lưu vào bộ đệm.
        \item In phản hồi ra màn hình.
    \end{enumerate}
    
    \item \textbf{Tham số truyền vào:} 
    \begin{itemize}
        \item \texttt{ConnectSocket}: Socket kết nối đến server.
        \item \texttt{recvbuf}: Bộ đệm để nhận dữ liệu.
        \item \texttt{recvbuflen}: Kích thước bộ đệm.
    \end{itemize}
    
    \item \textbf{Tham số trả về:} Chuỗi phản hồi từ server.
\end{itemize}

\paragraph{\textbf{Vòng lặp giao tiếp}}
\begin{itemize}
    \item \textbf{Prototype:} \texttt{void chatLoop(SOCKET\& ConnectSocket, char* recvbuf, int recvbuflen);}
    
    \item \textbf{Mục đích:} Thực hiện vòng lặp giao tiếp liên tục giữa client và server.
    
    \item \textbf{Cách thức hoạt động:} 
    \begin{enumerate}
        \item Lấy yêu cầu từ email (tích hợp với Gmail API).
        \item Gửi yêu cầu đến server bằng hàm \texttt{sendClientRequest()}.
        \item Nhận phản hồi từ server:
        \begin{itemize}
            \item Nếu phản hồi là tệp, sử dụng hàm \texttt{handleFileResponse()} để lưu tệp.
            \item Nếu phản hồi là văn bản, sử dụng hàm \texttt{handleTextResponse()} để hiển thị.
        \end{itemize}
        \item Gửi phản hồi lại email người gửi thông qua Gmail API.
        \item Lặp lại sau mỗi 10 giây hoặc thoát nếu yêu cầu là \texttt{"exit"}.
    \end{enumerate}
    
    \item \textbf{Tham số truyền vào:} 
    \begin{itemize}
        \item \texttt{ConnectSocket}: Socket kết nối đến server.
        \item \texttt{recvbuf}: Bộ đệm để nhận dữ liệu.
        \item \texttt{recvbuflen}: Kích thước bộ đệm.
    \end{itemize}
    
    \item \textbf{Tham số trả về:} Không có.
\end{itemize}

\paragraph{\textbf{Dọn dẹp tài nguyên}}
\begin{itemize}
    \item \textbf{Prototype:} \texttt{void cleanup(SOCKET\& ConnectSocket);}
    
    \item \textbf{Mục đích:} Giải phóng tài nguyên socket và thư viện Winsock.
    
    \item \textbf{Cách thức hoạt động:} 
    \begin{enumerate}
        \item Đóng socket bằng \texttt{closesocket()}.
        \item Gọi \texttt{WSACleanup()} để giải phóng Winsock.
    \end{enumerate}
    
    \item \textbf{Tham số truyền vào:} 
    \begin{itemize}
        \item \texttt{ConnectSocket}: Socket đã sử dụng.
    \end{itemize}
    
    \item \textbf{Tham số trả về:} Không có.
\end{itemize}

\paragraph{\textbf{In thanh tiến trình}}
\begin{itemize}
    \item \textbf{Prototype:} \texttt{void printProgressBar(int progress, int total);}
    
    \item \textbf{Mục đích:} In thanh tiến trình lên màn hình để hiển thị tỷ lệ hoàn thành công việc.
    
    \item \textbf{Cách thức hoạt động:} 
    \begin{enumerate}
        \item Tính toán tỷ lệ phần trăm dựa trên tiến độ (\texttt{progress}) và tổng số công việc (\texttt{total}).
        \item Xử lý việc hiển thị thanh tiến trình với chiều rộng cố định.
        \item Hiển thị ký tự \texttt{=} cho các phần đã hoàn thành và ký tự \texttt{>} cho phần hiện tại.
        \item Cập nhật thanh tiến trình liên tục trong quá trình công việc diễn ra.
    \end{enumerate}
    \newpage
    \item \textbf{Tham số truyền vào:} 
    \begin{itemize}
        \item \texttt{progress}: Số lượng công việc đã hoàn thành.
        \item \texttt{total}: Tổng số công việc cần hoàn thành.
    \end{itemize}
    
    \item \textbf{Tham số trả về:} Không có.
\end{itemize}

% Server
\subsubsection{Server}

\paragraph{\textbf{Khởi tạo Winsock}}
\begin{itemize}
    \item \textbf{Prototype:} \texttt{bool initializeWinsock(WSADATA\& wsaData);}
    
    \item \textbf{Mục đích:} Khởi tạo thư viện Winsock để sử dụng các API mạng.
    
    \item \textbf{Cách thức hoạt động:} 
    \begin{enumerate}
        \item Gọi hàm \texttt{WSAStartup()} để khởi tạo Winsock.
        \item Kiểm tra lỗi nếu việc khởi tạo thất bại.
    \end{enumerate}
    
    \item \textbf{Tham số truyền vào:} 
    \begin{itemize}
        \item \texttt{WSADATA\& wsaData}: Một cấu trúc chứa thông tin về phiên bản Winsock được khởi tạo.
    \end{itemize}
    
    \item \textbf{Tham số trả về:} \texttt{true} nếu khởi tạo thành công, \texttt{false} nếu thất bại.
\end{itemize}

\paragraph{\textbf{Cấu hình socket}}
\begin{itemize}
    \item \textbf{Prototype:} \texttt{bool setupSocket(SOCKET\& ListenSocket, addrinfo*\& result);}
    
    \item \textbf{Mục đích:} Tạo và cấu hình socket để chấp nhận kết nối từ client.
    
    \item \textbf{Cách thức hoạt động:} 
    \begin{enumerate}
        \item Thiết lập thông tin cấu hình cho socket sử dụng \texttt{addrinfo}.
        \item Gọi hàm \texttt{getaddrinfo()} để lấy thông tin cấu hình.
        \item Tạo socket bằng \texttt{socket()}.
        \item Gắn socket với địa chỉ và cổng bằng \texttt{bind()}.
        \item Bắt đầu lắng nghe kết nối bằng \texttt{listen()}.
    \end{enumerate}
    
    \item \textbf{Tham số truyền vào:} 
    \begin{itemize}
        \item \texttt{SOCKET\& ListenSocket}: Biến tham chiếu đại diện cho socket server.
        \item \texttt{addrinfo*\& result}: Con trỏ lưu trữ thông tin cấu hình socket.
    \end{itemize}
    
    \item \textbf{Tham số trả về:} \texttt{true} nếu cấu hình thành công, \texttt{false} nếu thất bại.
\end{itemize}

\paragraph{\textbf{In thông tin địa chỉ IP}}
\begin{itemize}
    \item \textbf{Prototype:} \texttt{void printIPAddress(const sockaddr\_in\& addr);}
    
    \item \textbf{Mục đích:} Hiển thị địa chỉ IP và cổng mà server đang lắng nghe.
    
    \item \textbf{Cách thức hoạt động:} 
    \begin{enumerate}
        \item Sử dụng lệnh \texttt{ipconfig} để lấy thông tin địa chỉ IP.
        \item Phân tích dữ liệu đầu ra để tìm địa chỉ IPv4.
        \item Hiển thị thông tin địa chỉ IP và cổng.
    \end{enumerate}
    
    \item \textbf{Tham số truyền vào:} 
    \begin{itemize}
        \item \texttt{const sockaddr\_in\& addr}: Cấu trúc chứa thông tin địa chỉ IP và cổng.
    \end{itemize}
    
    \item \textbf{Tham số trả về:} Không có.
\end{itemize}

\paragraph{\textbf{Hiển thị thông tin lắng nghe của socket}}
\begin{itemize}
    \item \textbf{Prototype:} \texttt{void printListeningInfo(SOCKET\& ListenSocket);}
    
    \item \textbf{Mục đích:} Hiển thị thông tin về địa chỉ IP và cổng mà server đang lắng nghe.
    
    \item \textbf{Cách thức hoạt động:} 
    \begin{enumerate}
        \item Lấy thông tin địa chỉ liên kết với socket bằng \texttt{getsockname()}.
        \item Chuyển đổi địa chỉ IP từ dạng nhị phân sang dạng chuỗi bằng \texttt{inet\_ntop()}.
        \item In thông tin địa chỉ IP và cổng ra màn hình.
    \end{enumerate}
    
    \item \textbf{Tham số truyền vào:} 
    \begin{itemize}
        \item \texttt{SOCKET\& ListenSocket}: Biến tham chiếu đại diện cho socket đang lắng nghe kết nối.
    \end{itemize}
    
    \item \textbf{Tham số trả về:} Không có.
\end{itemize}

\paragraph{\textbf{Chấp nhận kết nối từ client}}
\begin{itemize}
    \item \textbf{Prototype:} \texttt{SOCKET acceptClient(SOCKET\& ListenSocket);}
    
    \item \textbf{Mục đích:} Chấp nhận kết nối từ một client.
    
    \item \textbf{Cách thức hoạt động:} 
    \begin{enumerate}
        \item Gọi hàm \texttt{accept()} để chấp nhận kết nối từ client.
        \item Kiểm tra lỗi nếu việc chấp nhận kết nối thất bại.
    \end{enumerate}
    
    \item \textbf{Tham số truyền vào:} 
    \begin{itemize}
        \item \texttt{SOCKET\& ListenSocket}: Biến tham chiếu đại diện cho socket server đang lắng nghe.
    \end{itemize}
    
    \item \textbf{Tham số trả về:} Socket đại diện cho kết nối với client hoặc \texttt{INVALID\_SOCKET} nếu thất bại.
\end{itemize}

\paragraph{\textbf{Xử lý yêu cầu từ client}}
\begin{itemize}
    \item \textbf{Prototype:} \texttt{void processRequest(SOCKET\& ClientSocket, std::string\& request);}
    
    \item \textbf{Mục đích:} Xử lý yêu cầu từ client và trả về phản hồi phù hợp.
    
    \item \textbf{Cách thức hoạt động:} 
    \begin{enumerate}
        \item Loại bỏ ký tự không mong muốn khỏi yêu cầu.
        \item Phân tích cú pháp yêu cầu để xác định lệnh và tham số.
        \item Gọi hàm xử lý tương ứng dựa trên lệnh nhận được.
        \item Gửi phản hồi hoặc tệp tin đến client.
    \end{enumerate}
    
    \item \textbf{Tham số truyền vào:} 
    \begin{itemize}
        \item \texttt{SOCKET\& ClientSocket}: Biến tham chiếu đại diện cho kết nối với client.
        \item \texttt{std::string\& request}: Chuỗi chứa yêu cầu từ client.
    \end{itemize}
    
    \item \textbf{Tham số trả về:} Không có.
\end{itemize}

\paragraph{\textbf{Gửi tệp tin}}
\begin{itemize}
    \item \textbf{Prototype:} \texttt{void sendFile(SOCKET\& ClientSocket, std::string command);}
    
    \item \textbf{Mục đích:} Gửi tệp tin yêu cầu từ server đến client.
    
    \item \textbf{Cách thức hoạt động:} 
    \begin{enumerate}
        \item Xác định loại phản hồi và gửi thông tin này đến client.
        \item Lấy tên tệp tin từ lệnh yêu cầu.
        \item Gửi tên tệp, kích thước tệp và dữ liệu tệp đến client.
        \item Sử dụng \texttt{TransmitFile()} để gửi dữ liệu tệp.
    \end{enumerate}
    
    \item \textbf{Tham số truyền vào:} 
    \begin{itemize}
        \item \texttt{SOCKET\& ClientSocket}: Biến tham chiếu đại diện cho kết nối với client.
        \item \texttt{std::string command}: Lệnh chứa tên tệp tin cần gửi.
    \end{itemize}
    
    \item \textbf{Tham số trả về:} Không có.
\end{itemize}

\paragraph{\textbf{Gửi phản hồi}}
\begin{itemize}
    \item \textbf{Prototype:} \texttt{void sendResponse(SOCKET\& ClientSocket, std::string\& response);}
    
    \item \textbf{Mục đích:} Gửi phản hồi dạng văn bản đến client.
    
    \item \textbf{Cách thức hoạt động:} 
    \begin{enumerate}
        \item Xác định loại phản hồi là văn bản.
        \item Gửi kích thước và nội dung phản hồi đến client.
        \item Xử lý lỗi nếu quá trình gửi thất bại.
    \end{enumerate}
    
    \item \textbf{Tham số truyền vào:} 
    \begin{itemize}
        \item \texttt{SOCKET\& ClientSocket}: Biến tham chiếu đại diện cho kết nối với client.
        \item \texttt{std::string\& response}: Chuỗi chứa phản hồi cần gửi.
    \end{itemize}
    
    \item \textbf{Tham số trả về:} Không có.
\end{itemize}

\paragraph{\textbf{Xử lý kết nối từ client}}
\begin{itemize}
    \item \textbf{Prototype:} \texttt{void handleClient(SOCKET\& ClientSocket);}
    
    \item \textbf{Mục đích:} Quản lý và xử lý kết nối với một client.
    
    \item \textbf{Cách thức hoạt động:} 
    \begin{enumerate}
        \item Lắng nghe dữ liệu từ client.
        \item Phân tích và xử lý yêu cầu từ client.
        \item Đóng kết nối khi client gửi yêu cầu \texttt{exit} hoặc khi lỗi xảy ra.
    \end{enumerate}
    
    \item \textbf{Tham số truyền vào:} 
    \begin{itemize}
        \item \texttt{SOCKET\& ClientSocket}: Biến tham chiếu đại diện cho kết nối với client.
    \end{itemize}
    
    \item \textbf{Tham số trả về:} Không có.
\end{itemize}

\paragraph{\textbf{Dọn dẹp tài nguyên}}
\begin{itemize}
    \item \textbf{Prototype:} \texttt{void cleanup(SOCKET\& ClientSocket);}
    
    \item \textbf{Mục đích:} Dọn dẹp tài nguyên liên quan đến socket và Winsock.
    
    \item \textbf{Cách thức hoạt động:} 
    \begin{enumerate}
        \item Đóng socket sử dụng \texttt{closesocket()}.
        \item Giải phóng thư viện Winsock bằng \texttt{WSACleanup()}.
    \end{enumerate}
    
    \item \textbf{Tham số truyền vào:} 
    \begin{itemize}
        \item \texttt{SOCKET\& ClientSocket}: Biến tham chiếu đại diện cho kết nối với client.
    \end{itemize}
    \item \textbf{Tham số trả về:} Không có.
\end{itemize}

\paragraph{\textbf{Xử lý lệnh từ client}}
\begin{itemize}
    \item \textbf{Prototype:} \texttt{static void handleRequest(const std::string\& request, std::string\& response, std::string\& params, std::ifstream\& file);}
    \item \textbf{Mục đích:} Xử lý yêu cầu từ client và trả về phản hồi phù hợp.
    \item \textbf{Cách thức hoạt động:}
          \begin{enumerate}
            \item Phân tích yêu cầu để xác định lệnh và tham số.
            \item Gọi hàm xử lý tương ứng dựa trên lệnh nhận được.
            \item Ghi phản hồi vào chuỗi \texttt{response}.
            \item Xử lý lỗi nếu có trong quá trình xử lý yêu cầu.
          \end{enumerate}
    \item \textbf{Tham số truyền vào:}
    \begin{itemize}
        \item \texttt{const std::string\& request}: Chuỗi yêu cầu từ client.
        \item \texttt{std::string\& response}: Chuỗi chứa phản hồi trả về client.
        \item \texttt{std::string\& params}: Chuỗi chứa tham số từ yêu cầu.
        \item \texttt{std::ifstream\& file}: Đối tượng file để đọc dữ liệu từ tệp tin.
    \end{itemize}
    \item \textbf{Tham số trả về:} Không có.
\end{itemize}